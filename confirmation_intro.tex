\section*{Surrey PhD Confirmation Process}
Date: November 2015 \\
Student: Adam Thornton \\

\subsection*{PhD thesis: Radiation Effects and Test Requirements for High-Energy Mixed-Field Applications}

The doctoral project is focused on the study of radiation effect in electronics and to test for these effects, particularly in mixed-field radiation environments, such as those found in accelerators or in upper-atmosphere for example. Testing devices in the real environment can be difficult, and it would be much more practical if it were possible to test in a simulated environment with a large ‘acceleration factor’, where years of operation could be fit in to a time period orders of magnitude shorter. The challenge of this however is: how can these real environments be simulated? For this a metric is needed to compare the different environments, and a suitable technique to measure this metric. One of the main aims of the doctoral project is to define a metric and a measurement technique that can be applied to mixed-field radiation environments. \\

The work performed thus far has been directed towards completing a comprehensive study of the mixed radiation field inside the CHARM facility test area. This was achieved using the FLUKA Monte Carlo transport code, and programmed with the Flair editor for FLUKA. A report summarising this analysis has been written, which serves as a guide to the radiation environment within the test facility. In addition to the report as a reference, a website was created in order to give access to the processed data from the FLUKA calculation for each test positions. Using these tools it’s possible to get information about particle spectra and useful values (such as estimated dose and hadron fluence) at each test location. \\

In addition to the work performed directly towards the doctoral thesis, a number of contributions have been made towards papers submitted as conference proceedings (listed below). This consisted mostly of the analysis from the CHARM test area FLUKA studies (for the works presented at RADECS 2015). Among the help given to the various users testing at the CHARM facility over its first year, a large contribution in particular was given during the ‘single event latch-up’ (SEL) testing of some SRAM memories for which the data analysis was used for study presented at the NSREC 2014 conference. During these tests a potential metric for describing mixed-fields was analysed, and the results could potentially be included in the thesis. \\

A small separate analysis involving some data analysis of some Cypress brand SRAM tests was made for the paper submitted to NSREC 2014 conference. This contribution focused on the identification and analysis of burst events during the SRAM testing at PSI earlier that year. \\

\subsection*{List of Conference Proceedings Contributions}

\begin{itemize}
\item Qualification and Characterization of SRAM Memories used as Radiation Sensors in the LHC, IEEE TNS (NSREC 2014) \cite{6949151}
\item SEL Hardness Assurance in a Mixed Radiation Field, IEEE TNS (RADECS 2015) \cite{7312510}
\item CHARM - A Mixed Field Facility at CERN for Radiation Tests in Ground, Atmospheric, Space and Accelerator Representative Environments, IEEE TNS (RADECS 2015)
\item Distributed Optical Fiber Radiation Sensing at CERN High energy AcceleRator Mixed field facility (CHARM), IEEE TNS (RADECS 2015) \cite{7155495}
\end{itemize}

\subsection*{Timeline}
	
\begin{table}[htbp]
\centering
\begin{tabular}{r r l}
  \toprule
  \textbf{2014} & October & Start of the PhD \\
   & & CHARM Facility operational with first users \\
   & December & FLUKA Advanced Course \\
   & &\\
  \textbf{2015} & March & Interim Review (6 months) \\
   & November & Confirmation Process \\
   & & CHARM tests end for year \\
   & & \\  
  \textbf{2016} & March & Interim Review (18 months) \\
   & April & CHARM re-starts testing \\
   & October & Interim Review (24 months) \\
   & & \\
   \textbf{2017} & April & First draft of PhD thesis \\
   & & Interim Review (30 months) \\
   & October & Submit thesis! \\
  \bottomrule
\end{tabular}
\caption{Timeline of major events planned for the duration of the PhD project.}
\label{table:timeline}
\end{table}

