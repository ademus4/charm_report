%\documentclass[main.tex]{subfiles}
%\begin{document}

\newpage
\section{Measurements and Testing}
This chapter will summarise a number of measurements performed at CHARM, with the aim of later benchmarking the FLUKA calculated data, so that in the future we can rely confidently on the calculations, and minimise the number of measurements needed. There are a number of key areas of interest for the measurements which can be divided into the following; non-shielded test-area positions, shielded test-area positions, and the in-beam Montrac position \cite{charmblown} \\

A series of measurements were performed during 2014 and 2015 \cite{charmcalibration} using the Radmon \cite{Wijnands_radmon} detector system for dose and high-energy hadron equivalent fluence at a number of different test positions and facility configurations. A comparison has been made for the integral dose and fluence values calculated using FLUKA for respective positions, and the results for which are shown in tables \ref{tab:datatable-cpOOOO-dose}, \ref{tab:datatable-cpOOOO-heheq}, \ref{tab:datatable-cpCIIC-dose} and \ref{tab:datatable-cpCIIC-heheq}, where M/F stands for measurement divided by the value calculated with FLUKA. All the Radmon dose and HEHeq data was normalised per proton on target using the SEC1 with the calibration factor of 1.84E7 protons per count (or 554242 SEC1 counts) which based on aluminium foil activation \cite{pozzi2015}. The errors on the Radmon measurements can be assumed around 20\%. All values are an average of 3 measurements taken over 3 different test periods. \\

There is a systematic overestimate of the dose for both shielding configurations. This has been found to be due to the difference between scoring in air in the calculations compared to measuring the dose in silicon using the Radmon. Another contribution to the difference has been found in which thresholds are used when scoring the dose in air or silicon. As the thresholds are lowered, the dose scored in FLUKA reduces and becomes much more compatible with the measurements. A more detailed study on this is currently being made \cite{matteo2016}. \\

In the case without shielding. the calculations seem able to accurately model the high-energy hadron equivalent fluence, with the ratio between measurement and calculations around 1 in table \ref{tab:datatable-cpOOOO-heheq}. One exception to this is at position 10, however this value is based on only 2 measurements. For the case with shielding, the match is less strong, showing around a 20\% overestimate of the calculations. This may be due to a difference in the way the high-energy hadron equivalent fluence is modelled in FLUKA, as the measurements were made using the V6 Radmon and the original implementation was based on the V5 Radmon system. This has a particular impact on the 'intermediate' energy neutrons, for which there is a significantly different response between the 2 detectors. \\

\begin{table}[htbp]
  \centering
    \begin{tabular}{c|r|r|r}
    \textbf{Position} & \textbf{Radmon} & \textbf{FLUKA} & \textbf{M/F} \\
    \hline
    \hline
    1     & 6.35E-15 & 1.10E-14 & 0.69 \\
    2     & 6.85E-15 & 1.21E-14 & 0.68 \\
    3     & 1.26E-14 & 2.28E-14 & 0.66 \\
    5     & 9.26E-15 & 2.11E-14 & 0.53 \\
    7     & 1.18E-14 & 2.26E-14 & 0.63 \\
    9     & 1.11E-14 & 2.07E-14 & 0.64 \\
    10    & 1.36E-14 & 2.40E-14 & 0.68 \\
    \end{tabular}%
    \caption{A table of dose measurements showing the comparison with the FLUKA calculations at various test positions for the copper target (no shielding) configuration. The units for the dose are given in Gy per primary particle. }
  \label{tab:datatable-cpOOOO-dose}%
\end{table}%

\begin{table}[htbp]
  \centering
    \begin{tabular}{c|r|r|r}
    \textbf{Position} & \textbf{Radmon} & \textbf{FLUKA} & \textbf{M/F} \\
    \hline
    \hline
    1     & 2.70E-05 & 3.35E-05 & 0.97 \\
    2     & 2.30E-05 & 3.49E-05 & 0.79 \\
    3     & 4.02E-05 & 5.02E-05 & 0.96 \\
    5     & 3.62E-05 & 4.34E-05 & 1.00 \\
    7     & 3.92E-05 & 4.56E-05 & 1.03 \\
    9     & 3.70E-05 & 4.17E-05 & 1.06 \\
    10    & 5.69E-05 & 4.86E-05 & 1.41 \\
    \end{tabular}%
    \caption{A table of HEHeq measurements showing the comparison with the FLUKA calculations at various test positions for the copper target (no shielding) configuration. The units of HEHeq are given per centimetre squared, per primary particle. }
  \label{tab:datatable-cpOOOO-heheq}%
\end{table}%

\begin{table}[htbp]
  \centering
    \begin{tabular}{c|r|r|r}
    \textbf{Position} & \textbf{Radmon} & \textbf{FLUKA} & \textbf{M/F} \\
    \hline
    \hline
    1     & 4.67E-16 & 6.78E-16 & 0.83 \\
    2     & 4.82E-16 & 7.57E-16 & 0.76 \\
    3     & 7.60E-16 & 1.30E-15 & 0.70 \\
    5     & 1.00E-15 & 1.74E-15 & 0.69 \\
    7     & 1.18E-15 & 2.03E-15 & 0.70 \\
    9     & 1.18E-15 & 2.74E-15 & 0.52 \\
    \end{tabular}%
    \caption{A table of dose measurements showing the comparison with the FLUKA calculations at various test positions for the copper target (full shielding) configuration. The units for the dose are given in Gy per primary particle. }
  \label{tab:datatable-cpCIIC-dose}%
\end{table}%

\begin{table}[htbp]
  \centering
    \begin{tabular}{c|r|r|r}
    \textbf{Position} & \textbf{Radmon} & \textbf{FLUKA} & \textbf{M/F} \\
    \hline
    \hline
    1     & 1.34E-06 & 1.86E-06 & 0.87 \\
    2     & 1.51E-06 & 2.39E-06 & 0.76 \\
    3     & 2.74E-06 & 4.54E-06 & 0.72 \\
    5     & 2.94E-06 & 5.13E-06 & 0.69 \\
    7     & 3.45E-06 & 5.55E-06 & 0.75 \\
    9     & 4.36E-06 & 6.08E-06 & 0.86 \\
    \end{tabular}%
    \caption{A table of HEHeq measurements showing the comparison with the FLUKA calculations at various test positions for the copper target (full shielding) configuration. The units of HEHeq are given per centimetre squared, per primary particle.}
  \label{tab:datatable-cpCIIC-heheq}%
\end{table}%